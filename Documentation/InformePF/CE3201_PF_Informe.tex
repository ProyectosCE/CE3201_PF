\documentclass[conference]{IEEEtran}
\IEEEoverridecommandlockouts
\usepackage{cite}
\usepackage{amsmath,amssymb,amsfonts}
\usepackage{algorithmic}
\usepackage{graphicx}
\usepackage{textcomp}
\usepackage{float}
\usepackage{xcolor}
\usepackage{booktabs}  % Si deseas conservar toprule, midrule y bottomrule
\usepackage{tabularx}  % Para usar el entorno tabularx

\usepackage{listings}

\usepackage[backend=bibtex,style=ieee,biblabel=dot,language=spanish]{biblatex}
\addbibresource{references.bib}

\usepackage[pdftex,
            pdftitle={Lógica Combinacional y Aritmética}, % Título del PDF
            pdfauthor={Barquero, J., Campos, J., Feng, J., y Montero, A.},   % Autor delPDF
            pdfsubject={Taller de Diseño Digital - CE3201}, % Tema del PDF
            pdfkeywords={}, % Palabras clave del PDF
            pdfproducer={LaTeX with hyperref package},
            pdfcreator={pdflatex}]{hyperref}

\hypersetup{
    colorlinks=true,        % Colorea los enlaces en lugar de usar cajas alrededor de ellos
    linkcolor = black,
    urlcolor  = blue,
    citecolor = black,
    anchorcolor = black         % Color de los enlaces externos
}

\renewcommand{\abstractname}{Resumen}


\lstset{
    language=VHDL,
    basicstyle=\small\ttfamily,
    breaklines=true,    % Ajusta automáticamente las líneas
    keywordstyle=\color{blue},
    commentstyle=\color{gray}
}

\renewcommand{\tablename}{Tabla}
\def\BibTeX{{\rm B\kern-.05em{\sc i\kern-.025em b}\kern-.08em
    T\kern-.1667em\lower.7ex\hbox{E}\kern-.125emX}}
\begin{document}

\title{\textbf{Lógica Combinacional y Aritmética}}

\author{
    \IEEEauthorblockN{José Bernardo Barquero Bonilla}
    \IEEEauthorblockA{
        Carné: 2023150476 \\
        Correo: jos.barquero@estudiantec.cr \\
        Escuela de Ingeniería en Computadores \\
        Instituto Tecnológico de Costa Rica
    }
    \and
    \IEEEauthorblockN{Jose Eduardo Campos Salazar}
    \IEEEauthorblockA{
        Carné: 2023135620 \\
        Correo: j.campos@estudiantec.cr \\
        Escuela de Ingeniería en Computadores \\
        Instituto Tecnológico de Costa Rica
    }
    \and
    \IEEEauthorblockN{Jimmy Feng Feng}
    \IEEEauthorblockA{
        Carné: 2023060374 \\
        Correo: jifeng@estudiantec.cr \\
        Escuela de Ingeniería en Computadores \\
        Instituto Tecnológico de Costa Rica
    }
    \and
    \IEEEauthorblockN{Alexander Montero Vargas}
    \IEEEauthorblockA{
        Carné: 20233166058 \\
        Correo: ale\_montero@estudiantec.cr \\
        Escuela de Ingeniería en Computadores \\
        Instituto Tecnológico de Costa Rica
    }
}

\maketitle

\begin{abstract}
Aquí va el abstract
\end{abstract}

\renewcommand\IEEEkeywordsname{Palabras clave}
\begin{IEEEkeywords}  
  Palabras, clave, del, informe, aquí
\end{IEEEkeywords}

\section{Introducci\'on}
Introduccion
\section{Marco Teórico}
Marco teórico
\section{Desarollo}
Desarollo

\section{Análisis de Resultados}
Análisis de resultados
\section{Conclusiones}
Conclusiones

\nocite{*}
\printbibliography

\end{document}
